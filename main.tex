\documentclass[twocolumn,10pt]{article}
\usepackage[utf8]{inputenc}
\usepackage{CJKutf8}
\usepackage{graphicx}
\usepackage{listings}
\usepackage{hyperref}
\usepackage{indentfirst}

\hypersetup{
    colorlinks=true,
    linkcolor=blue,
    filecolor=magenta,      
    urlcolor=cyan,
    pdftitle={Overleaf Example},
    pdfpagemode=FullScreen,
    }
\usepackage[a4paper, top = 2.5cm, bottom = 2.5cm, left = 3.17cm, right = 3.17cm]{geometry}
\setlength{\parindent}{12pt}

\title{zero-shot nas}
\author{謝承翰}
\date{2023}

\begin{document}
\begin{CJK*}{UTF8}{bsmi}

\maketitle

\section{摘要}
近年來,人造神經網路(Artificial neural network; ANN)在各領域中有廣泛的運用,例如: 影像辨識、影音辨識,以及自然語言處理(Natural Language Processing; NLP)。\par
過去大部分的人造神經網路架構,都是根據經驗手動尋找、調整,所需的資源、時間過多,研究如何自動搜尋神經網路架構(Neural Architecture Search; NAS)的重要性因而上升。\par
但以往的神經網路架構搜尋演算法,在尋找的過程中,必須訓練一部分神經網路後才能得知架構的好壞,耗費許多時間、資源,故近年許多Traning-free的神經架構搜尋演算法不斷被提出,Traning-free演算法使用Training-Free Score Function評價神經網路架構,而不需訓練,相較以往,Traning-free的演算法更迅速、耗費的資源更少。\par
然而Training-Free Score Function與神經網路訓練後的準確度相關性不高,搜尋的結果欠佳,本研究會結合三種不同的Training-Free Score Function,搭配基因演算法(Genetic Algorithm; GA)搜尋架構,並在NAS-Bench-101, NAS-Bench-201等可復現的架構集驗證結果。預計提出一個速度、準確度兼顧的神經網路架構搜尋演算法。\par

\section{參考文獻}
\begin{thebibliography}{9}
\end{thebibliography}


\end{CJK*}
\end{document}
